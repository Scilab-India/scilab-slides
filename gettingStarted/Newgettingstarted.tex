% { ( [
\documentclass[brown]{beamer}[17pt] 
\setbeamercolor{alerted text}{fg=brown}
\usepackage{verbatim}
\usepackage{bm}
\newenvironment{colorverbatim}[1][]
{
\color{blue}
}
{
\endverbatim
}
\usepackage{beamerthemesplit}
\usepackage{graphicx}
\usepackage{eso-pic}
\usepackage{beamerthemeshadow}
\beamertemplateshadingbackground{blue!5}{yellow!10}
\usepackage{beamerthemesplit}
\logo{\includegraphics[height=1cm]{3t-logo.pdf}}

\begin{document}
%\sffamily 
\bfseries
\title
[ Introduction to Xcos
\hspace{0.5cm}
\insertframenumber/\inserttotalframenumber]
{\LARGE Introduction to Xcos}
\author[Script-Shalini Shrivastava\\  Narration-Anuradha Amrutkar]{{\Large Talk to a Teacher \\ National Mission on Education through ICT \\ http://spoken-tutorial.org }\\ [0.5cm]	{\scriptsize Script}\\Shalini Shrivastava \\{\scriptsize IIT Bombay } \\{\scriptsize Narration} \\  Anuradha Amrutkar \\{\scriptsize IIT Bombay}\vspace{-.5 cm}}
 \date{13 July 2011}

\begin{frame}
   \titlepage
\end{frame}

\begin{frame}[fragile]
\frametitle{Objectives }
At the end of this tutorial, you will learn:
\begin{itemize}
\item What is XCOS
\item What is palette
\item To collect the blocks from the palette and connect them to construct the block diagram.
\item Set the parameters of different block 
\item To setup the simulation parameters
\item Simulate the constructed block diagram
\end{itemize}
\end{frame}

\begin{frame}[fragile]
\frametitle{Prerequisites}
\begin{itemize}
\item Scilab should be installed.
\end{itemize}
\end{frame}


\begin{frame}[fragile]
\frametitle{Summary}
In this tutorial we have learnt that:
\begin{itemize}
\item Xcos is scilab package for modelling and simulation of hybrid dynamic system models
\item A palette browser list all Xcos standard blocks grouped by categories
\item The block are collect in XCOS window by select and dram them from the selected palette
\item The parameter of each block is set by double click on the particular block
\item The setup for simulation is change by set up option in simulation available in menu bar of xcos
\item The data resulting of the simulation can then be graphically viewed in realtime in another window.
\end{itemize}
\end{frame}


\begin{frame}[fragile]
\frametitle{Acknowledgement}
\begin{itemize}
\item Spoken tutorials are part of Talk to a Teacher
\item Supported by the National Mission on Eduction through ICT, MHRD, Government of India
\item More information: \\
	{\color{magenta}http://spoken-tutorial.org/NMEICT-Intro.}

\end{itemize}
\end{frame}
\end{document}





