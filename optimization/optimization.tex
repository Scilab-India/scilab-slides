% { ( [
\documentclass{beamer}
% \documentclass[handout]{beamer}% to get rid of fancy frames, do \frame[plain]{
\usepackage{beamerthemesplit}
\usepackage{graphicx,graphics}



%------------------------------------------------------
\usepackage{latexsym,multicol}
\usepackage{epsfig,epsf,colordvi,color}

\usepackage{ulem}
\definecolor{bluecol}{rgb}{0,0,1}
\definecolor{blackcol}{rgb}{1,1,1}
\definecolor{redcol}{rgb}{1,0,0}
\newcommand{\blue}[1]{{\color{bluecol}#1}}
\newcommand{\black}[1]{{\color{blackcol}#1}}
\newcommand{\red}[1]{{\color{redcol}#1}}


%--------------------------------------------------------
\usetheme{Darmstadt}
% \setbeamercovered{transparent}
\title[Optimization] {Optimization in Scilab}
\date{27 Dec, 2011}
\author[Manas Das \& Anuradha Singhania]{Manas Das \& Anuradha Singhania}
\institute[IIT-Bombay]{
  \large Department of Electrical Engineering\\
  Indian Institute of Technology Bombay\\
  {FOSSEE}
}
%____________________________________________
\newcommand{\deri}[1]{\mbox{$\frac{{\mathrm d}^{#1}}{{\mathrm d}t^{#1}}$}}
\newcommand{\der}      {\deri{}}
%____________________________________________
\begin{document}
%--------------------------------------------
\frame[plain]{
\titlepage
}
%--------------------------------------------
\section*{Outline}
\frame{
\frametitle{Outline}
\tableofcontents
}
%--------------------------------------------------------------
\section{Scilab}
\frame{ \frametitle{Today's focus}
\begin{itemize}
\item Scilab is free. \pause % Matlab costs a lakh (per educational license). \pause
\item Matrix/loops syntax is similar to Matlab. \pause
\item Scilab provides all basic and many advanced tools. \pause
\item Today: Optimization in Scilab \alert{optim command}
\end{itemize}
}
%----------------------------------------------------------------
\section{Optimization}

\frame{
\frametitle{Optimization}
\begin{itemize}
\item Minimize/maximize some (more than one) objective function by
varying decision variable
\item Subject to constraints on functions of decision variables
\item All engineering domains, economics  
\item Vast area in itself
\end{itemize}
}

\frame{ 
\frametitle{Single Objective Optimization Problems}
\begin{equation}
\min_{x}~~ f(x)
\end{equation}
where $x$ is $n$ dimensional, $f(x)$ is a scalar, with/without
\begin{itemize}
\item Bound constraints: $x_L \leq x \leq x_U$
\item Linear/Nonlinear equality constraints: $g(x)=b$
\item Linear/Nonlinear inequality constraints: $g(x)\leq b$
\end{itemize}
Components of $x$ can be real, integer or binary. \\
Find \alert{`Decision Variables'}: $x$ which minimize \alert{'Obj'}: $f$ 

}


\frame{
\frametitle{Linear optimization with Linear constraints}
\begin{itemize}
\item Scilab function: \bf {karmarkar} \\
Calling Sequence \\
[x1]=\bf karmarkar(a,b,c,lb)
\begin{itemize}{\tt
\item a:matrix (n,p)
\item b:n-vector
\item c:p-vector
\item lb:lower bound
}
\end{itemize}
\item Also be solved in scilab using  linpro  function 
\item linpro is in Quapro toolbox under Optimization in ATOMS,
\item It can installed by command atomsInstall('Quapro')
\end{itemize}
}
\frame{
\frametitle{Linear optimization with Linear constraints}
Example:- \\
     Maximize \\
     $ 3x_1 + x_2 + 3x_3 \\
     for \\
     2x_1 + x_2 + x_3 \le 2 \\
     x_1 + 2x_2 + 3x_3 \le 5 \\
     2x_1 + 2x_2 + x_3 \le 6 \\
     x1, x2, x3 \ge 0 $\\
}
\frame{
\frametitle{Quadratic with linear constraints}
\begin{itemize}{\tt
\item Scilab function: \bf {qp\_solve} \\
\tt {Calling Sequence} \\
[x1]=\bf [x [,iact [,iter [,f]]]]=qp\_solve(Q,p1,C1,b,me)
\begin{itemize}{\tt
\item Q:real positive definite symmetric matrix (dimension n x n).
\itemp p- real (column) vector (dimension n)
\item c:p - vector
\item lb:lower bound
\item me: number of equality constraints
}
\end{itemize}
\item It can  also be solved in scilab using  quapro function.
\item quapro is in Quapro toolbox under Optimization in ATOMS
\item It can installed by command atomsInstall(quapro)
}
\end{itemize}
}
\frame{
\frametitle{Quadratic with linear constraints}
Example
    Minimize \\
     $ f(x) =  \frac{1}{2} x^T
     \begin{bmatrix}
         2 & 0 \\
         0 & 2 \\
     \end{bmatrix} 
     x + 
     \begin{bmatrix}
         -2 \\
         -2 \\
     \end{bmatrix} ^ T x \\
     for \\
     x_1 + x_2 \ge 2 + \sqrt {2} \\
     -x_1 + x_2 \ge -2 $ \\
}
\section{Other Optimization Tools in Scilab}
\frame{
\frametitle{Other Optimization Tools in Scilab: I}
Native (built-in) functions in Scilab:
\begin{itemize}
\item fminsearch: Computes unconstrained minimum of a given function
using the Nelder-Mead algorithm (simplex region search).
\begin{itemize}
\item Native (built-in function) to Scilab:
\item Derivative free, relies on search
\item Basic usage: [xopt,fopt]=fminsearch(costf,x0); costf should take
x as input and return f as output
\item Try minimizing $f(x)=x^2+5$ using fminsearch
\end{itemize}
\item neldermead: Computes minimum of a nonlinear function subject to
bounds and nonlinear constraints
\begin{itemize}
\item Option for several types of search algorithms
\end{itemize}
\end{itemize}
}
\frame{
\frametitle{Unconstrained  Non-linear}
\begin{itemize}{\tt
\item Scilab function: \bf {fminsearch} \\
\tt {Calling Sequence} \\
 x = fminsearch ( costf , x0 )
\begin{itemize}{\tt
\item costf:The cost function.
\itemp x0:The initial guess.
}
\end{itemize}
\item computes the unconstrained minimimum of given function with the Nelder-Mead algorithm.
}
\end{itemize}
}

\frame{
\frametitle{Unconstrained  Non-linear}
Example:
$ f(x) = x_1^2 + x_2^2 + x_1 x_2 $
}

\section{``Optim" in Scilab}
\frame{
\frametitle{Optim command in Scilab}
Simplified Call: 

[fopt,xopt]=optim(costf,x0) \\
where 
\begin{itemize}
\item $x0$: initial guess where minimum of f occurs, 
\item $fopt$: the optimum (minimum) value, 
\item $xopt$: where the optimum occurs, 
\item $costf$: user specified (scilab) function to compute objective
function (f) and/or its gradient (g) depending on the input integer flag
``ind" as: 

[f,g,ind]=costf(x,ind) 
\end{itemize}
}

\frame{
\frametitle{Simple example}
$f(x)=x^2+10$ \\
$g(x)=2x$ 

\begin{itemize}
\item Minimize $f(x)$ starting from $x0=10$.
\item Write a function (example ``myfunction1.sci") which takes $x,ind$
as input and returns $f,g,ind$ as output.
\item Use ``optim" function to minimize.
\item Answer: $fopt=10,xopt=0$. 
\end{itemize}
}


\frame{
\frametitle{Another example (Campbell et al., 2010) }
\begin{equation*}
f(x_1,x_2,x_3)=(x_1-x_3)^2 + 3(x_1+x_2+x_3-1)^2 + (x_1-x_3+1)^2
\end{equation*}
with gradient
\begin{eqnarray*}
g & = & [\frac{\partial f}{\partial x_1}, \frac{\partial
f}{\partial x_2},\frac{\partial f}{\partial x_3}] \\
& = & 
[2(x_1-x_3)+6(x_1+x_2+x_3-1)+2(x_1-x_3+1), \\
& & 6(x_1+x_2+x_3-1), -2(x_1-x_3)-2(x_1-x_3+1)+6(x_1+x_2+x_3-1)]
\end{eqnarray*}
}


\frame{
\frametitle{Exercise plan}
\begin{itemize}
\item Minimize the function starting with guess values $x0=[1,1,1]$.
\item Write a function (example ``myfunction2.sci") which takes $x,ind$
as input and returns $f,g,ind$ as output.
\item Use ``optim" function to minimize.
\item Answer: $xopt=[0.14,0.27,0.64],fopt=0.51$.
\end{itemize}

}



\frame{
\frametitle{Other options with optim}
Several other options/features with optim
\begin{itemize}
\item Several optimization algorithms: quasi-newton, conjugate
gradient, etc.
\item Gradient computation using finite differences: ``NDcost" used
with optim.
\item Various stopping/diagnostic criteria: Maximum number of
iterations, calls to costf, thresholds, etc.
\item Use of ind to compute gradient only when required (save
computational effort).
\item help optim: to know more
\end{itemize}
}




\frame{
\frametitle{Other Optimization Tools in Scilab: II}
Native (built-in) functions in Scilab: \\
AI based algorithms
\begin{itemize}
\item Genetic algorithms: single objective optimization,
multiobjective optimization, various implementations
\item Simulated annealing 
\end{itemize}
}


\frame{
\frametitle{Optimization Toolboxes in Scilab}
Can be installed easily (atoms or otherwise) or interfaces to them
\begin{itemize}
\item quapro: linear, quadratic programming ($Q$ not necessarily
positive definite)
\item IPopt, fsqp, lp\_solve, etc. 
\end{itemize}
Check ``http://atoms.scilab.org/categories/optimization"
}

\frame{
\frametitle{Optimization Features available in Scilab}

\begin{itemize}
\item Nonlinear optimization with the \textbf{optim} function
\item  Quadratic optimization with the  \textbf{qpsolve} function
\item  Nonlinear least-square optimization with the  \textbf{lsqrsolve} function
\item Semidefinite programming with the  \textbf{semidef} function
\item Genetic algorithms with the  \textbf{optim\_ga} function
\item Simulated annealing with the  \textbf{optim\_sa} function
\item Linear matrix inequalities with the  \textbf{lmisolver} function
\end{itemize}
}
\frame{
\frametitle{Missing optimization features in Scilab}
\texttt{A list of features which are not available in Scilab, but are available in toolboxes. }
\begin{itemize}
\item Integer parameter with linear objective solver and sparse matrices : currently available in
                 \textbf{LPSOLVE} toolbox, based on the simplex method,
\item Linear objective with sparse matrices : currently available in  \textbf{ LIPSOL}, based on interior
               points method,
\item Nonlinear objective and non linear constraints : currently available in interface to   \textbf{IPOPT}
               toolbox, based on interior point methods,
\item Nonlinear objective and non linear constraints : currently available in interface to  \textbf {CONMIN}
               toolbox, based on method of feasible directions,
\end{itemize}
}


\frame{
References
\begin{itemize}
\item S. L. Campbell, J.P. Chancelier and R. Nikoukhah, ``Modeling and
Simulation in Scilab/Scicos", Springer, 2006.
\end{itemize}
{\bf Thank You}
}



\end{document}
