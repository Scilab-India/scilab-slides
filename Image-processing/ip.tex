\documentclass[17pt]{beamer} 
\usepackage{amsmath}
\usepackage{hyperref}
\usepackage{epsfig}
%\usepackage{psfig}
\definecolor{Goldenrod}{RGB}{184,134,11}
\setbeamercolor{structure}{fg=Goldenrod}
\definecolor{blue}{RGB}{1,51,102}
\setbeamercolor{alerted text}{fg=blue}
%\beamertemplateshadingbackground{blue!5}{yellow!10}
\usepackage{beamerthemesplit}
\usepackage{graphicx}

\logo{\includegraphics[height=1.15cm]{3t-logo.pdf}}

\begin{document}

\sffamily \bfseries
\title
[Image processing in Scilab
\hspace{1cm}
\insertframenumber/\inserttotalframenumber]
{\normalsize Image processing in Scilab}
\author
[Abhishek Pawar]
{\small  Narration: Anuradha Amrutkar \\ Script: Abhishek Pawar\\ IIT Bombay \\ [0.25in]
Talk to a Teacher Project \\ http://spoken-tutorial.org \\
  National Mission on Education through ICT \\ 
 \today}
\date{}




\begin{frame}
\maketitle
\end{frame}

\begin{frame}
\frametitle{Learning Objectives}
In this tutorial we will learn:
\begin{itemize} [<+-|alert@+>]
\item How to install SIVP toolbox
\item Basic commands related to SIVP
\item Inserting noise in image
\item Creating filter for image
\end{itemize}
\end{frame}



\begin{frame}
\frametitle{System Requirement}
\begin{itemize}[<+-|alert@+>]
\item OS: Ubuntu 11.04 
\item Scilab 5.3.X
\end{itemize}
\end{frame}


\begin{frame}
\frametitle{Prerequisites}
\begin{itemize}[<+-|alert@+>]
\item Listen to basic Level tutorials in Scilab
\end{itemize}
\end{frame}

\begin{frame}
\frametitle{Toolbox}
\begin{itemize} [<+-|alert@+>]
\item There are two toolboxes,
\begin{itemize}
\item SIP
\item SIVP
\end{itemize}
\item SIVP stands for Scilab Image and Video Processing
\item Works on both windows and linux OS
\end{itemize}
\end{frame}

\begin{frame}
\frametitle{Summary}
In this tutorial we learnt,
\begin{itemize}[<+-|alert@+>]
\item Basic commands in Scilab(im2bw,rgb2gray etc)
\item Inserting noise in image
\item Creating filter for image
\end{itemize}
\end{frame}

\begin{frame}
\frametitle{Assignment}
\begin{enumerate}
\item Convert a given image into 32 bit signed integer\pause
\item Convert a given image into 16 bit unsigned integer\pause
\item Add 'speckle' noise to given image \pause
\item Resize a given image by factor of 1.5 using bilinear interpolation\pause
\end{enumerate}
\end{frame}


\begin{frame}
  \frametitle{About the Spoken Tutorial Project}
  \begin{itemize}[<+-|alert@+>]
  \item Watch the video available at {\color{blue} http://spoken-tutorial.org
    /What\_is\_a\_Spoken\_Tutorial}
  \item It summarises the Spoken Tutorial project
  \item If you do not have good bandwidth, you can download and watch it
  \end{itemize}
\end{frame}

\begin{frame}[<+-|alert@+>]
  \frametitle{About the Spoken Tutorial Project}
The Spoken Tutorial Project Team
  \begin{itemize}[<+-|alert@+>]
\item Conducts workshops using spoken tutorials
\item Gives certificates to those who pass an online test
\item For more details, contact \\
contact@spoken-tutorial.org
  \end{itemize}
\end{frame}

\begin{frame}
  \frametitle{Acknowledgement}
  \begin{itemize}[<+-|alert@+>]
  \item Spoken Tutorial Project is a part of the Talk to a Teacher project
  \item It is supported by the National Mission on Education through ICT, MHRD, Government of India
\item  More information on the same is available at:\\  { \color{blue} http://spoken-tutorial.org /NMEICT-Intro}
  \end{itemize}
\end{frame}


\begin{frame}
\frametitle{About the contributor}
%{\color{blue}Mumbai University}\\
\begin{itemize}[<+-|alert@+>]

\item This script is contributed by Abhishek Pawar
\item This is Anuradha Amrutkar signing off
\item Thank you
\end{itemize}
\end{frame}

\end{document}

