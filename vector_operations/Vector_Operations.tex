\documentclass[17pt]{beamer}
\usepackage{amsmath}
\usepackage{hyperref}
\usepackage{epsfig}
%\definecolor{Purple1}{RGB}{0,102,102}
\setbeamercolor{structure}{fg=brown}
%\definecolor{blue}{RGB}{1,51,102}

\setbeamercolor{alerted text}{fg=brown}
\usepackage{verbatim}
\usepackage{bm}
\usepackage{txfonts}
\newenvironment{colorverbatim}[1][]
{
\color{blue}
}
{
\endverbatim
}
\usepackage{beamerthemesplit}
\usepackage{graphicx}
\usepackage{eso-pic}
\usepackage{beamerthemeshadow}
\beamertemplateshadingbackground{blue!5}{yellow!10}
\usepackage{beamerthemesplit}
\logo{\includegraphics[height=1cm]{3t-logo.pdf}}

\begin{document}
\sffamily 
\bfseries
\title
[ Vector Operations
\hspace{0.5cm}
\insertframenumber/\inserttotalframenumber]
{\normalsize Vector Operations}
\vspace{-1 cm}
\author[ Script-Shalini \\ Narration-Anuradha]{{ \small Talk to a Teacher \\ National Mission on Education through ICT \\ http://spoken-tutorial.org }\\{\scriptsize Script }\\ \vspace{-0.2cm}{\small Shalini Shrivastava}\vspace{-0.3cm} \\{\scriptsize (IIT Bombay) } \vspace{-0.2cm}\\{\scriptsize Narration} \vspace{-0.2cm}\\  {\small Anuradha Amrutkar}\vspace{-0.3cm} \\{\scriptsize (IIT Bombay)}\vspace{-1 cm}}

\date{ \scriptsize 19 August 2011}

\begin{frame}
\maketitle
\end{frame}

\begin{frame}[fragile]
\frametitle{Objectives}
At the end of this spoken tutorial, you will be able to:\pause
\begin{itemize} [<+-|alert@+>]
\item Define a vector.
\item Calculate length of a vector.
\item Perform mathematical operations on vectors such as addition, subtraction and multiplication.

\end{itemize}
\end{frame}

\begin{frame}[fragile]
\frametitle{Objectives}
\begin{itemize} [<+-|alert@+>]
\item Define a matrix.
\item Calculate size of a matrix.
\item Perform mathematical operations on matrices such as addition, subtraction and multiplication.
\end{itemize}
\end{frame}



\begin{frame}[fragile]
\frametitle{Prerequisites}
\begin{itemize} [<+-|alert@+>]
\item Scilab Installed on your machine. 
\item Spoken Tutorial: Getting started with scilab.
\item *Basic knowledge about Vector and Matrix.
\item I am using Windows 7 OS and Scilab 5.2.2 for demonstration.
\end{itemize}
\end{frame}

\begin{frame}[fragile]
\frametitle{Exercise 1}
\begin{itemize}
\item Define two vectors A,B with 1,5,8,19 and 19,8,5,1 elements respectively.
\item Calculate A'*B-B'*A
\item Calculate A*A'+B*B'
\end{itemize}
\end{frame}

\begin{frame}[fragile]
\frametitle{Exercise 2}
In Scilab Console, enter the following\\
Matrices:
\textsl{A = $\left[
               \begin{array}{cc}
                 1 &  \frac{1}{2}   \\ \\
                 \frac{1}{3} &  \frac{1}{4}  \\ \\
                 \frac{1}{5} & \frac{1}{6} 
              \end{array}
             \right]$ }\\
B =[ 5 −2 ],
\textsl{C = $\left[
               \begin{array}{ccc}
                 4 &  \frac{5}{4}  & \frac{9}{4}   \\ \\
                 1 & 2 & 3  
                \end{array}
             \right]$ }\\
             
\end{frame} 

\begin{frame}[fragile]
\frametitle{Exercise 2}             
Compute each of the following and explain the errors, if any.
\begin{itemize}
\item $A*C-C*A$
\item $2*C-6*A$
\item $(2*C-6*A')*B'$
\item $(2*C-6*A')*C'$
\end{itemize}
\end{frame}  


\begin{frame}[fragile]
\frametitle{Summary}
In this tutorial, we have learnt to,
\begin{itemize}
\item Define a vector using spaces or commas.
\item Calculate length of a vector using the length() function.
\item Find the transpose of vector or matrix using apostrophe.
\end{itemize}
\end{frame}  

\begin{frame}[fragile]
\frametitle{Summary}
\begin{itemize}
\item Define a matrix by using space/comma to separate the columns and semicolon to separate the rows.
\item Find size of a matrix using size() function.
\end{itemize}
\end{frame}


\begin{frame}[fragile]
\frametitle{Acknowledgement}
\begin{itemize}
\item This spoken tutorial has been created by the Free and Open Source Software in Science and Engineering Education(FOSSEE).\pause
\item More information on the FOSSEE project could be obtained from {\color{magenta}http://fossee.in} or {\color{magenta}http://scilab.in}\pause
\end{itemize}
\end{frame}

\begin{frame}[fragile]
\frametitle{Acknowledgement}
\begin{itemize}
\item Supported by the National Mission on Eduction through ICT, MHRD, Government of India.\pause
\item For more information, visit: \\
	{\color{magenta}http://spoken-tutorial.org/NMEICT-Intro}

\end{itemize}
\end{frame}
\end{document}
