\documentclass[17pt]{beamer}
\usepackage{amsmath}
\usepackage{hyperref}
\usepackage{epsfig}
%\definecolor{Purple1}{RGB}{0,102,102}
\setbeamercolor{structure}{fg=brown}
%\definecolor{blue}{RGB}{1,51,102}

\setbeamercolor{alerted text}{fg=brown}
\usepackage{verbatim}
\usepackage{bm}
\usepackage{txfonts}
\newenvironment{colorverbatim}[1][]
{
\color{blue}
}
{
\endverbatim
}
\usepackage{beamerthemesplit}
\usepackage{graphicx}
\usepackage{eso-pic}
\usepackage{beamerthemeshadow}
\beamertemplateshadingbackground{blue!5}{yellow!10}
\usepackage{beamerthemesplit}
\logo{\includegraphics[height=1cm]{3t-logo.pdf}}

\begin{document}
\sffamily 
\bfseries
\title
[ Matrix Operations
\hspace{0.5cm}
\insertframenumber/\inserttotalframenumber]
{\normalsize Matrix Operations}
\vspace{-1 cm}
\author[ Script-Shalini \\ Narration-Anuradha]{{ \small Talk to a Teacher \\ National Mission on Education through ICT \\ http://spoken-tutorial.org }\\{\scriptsize Script }\\ \vspace{-0.2cm}{\small Shalini Shrivastava}\vspace{-0.3cm} \\{\scriptsize (IIT Bombay) } \vspace{-0.2cm}\\{\scriptsize Narration} \vspace{-0.2cm}\\  {\small Anuradha Amrutkar}\vspace{-0.3cm} \\{\scriptsize (IIT Bombay)}\vspace{-1 cm}}

\date{ \scriptsize 19 August 2011}

\begin{frame}
\maketitle
\end{frame}





\begin{frame}[fragile]
\frametitle{Objectives }
At the end of this spoken tutorial, you will be able to:
\begin{itemize}
\item Access the element of Matrix. \pause
\item Determine the determinant, inverse and eigenvalue of a matrix.\pause
\item Define the special matrices.
\end{itemize}
\end{frame}


\begin{frame}[fragile]
\frametitle{Objectives }
\begin{itemize}
\item Perform elementary row operations. \pause
\item Solve the system of linear equations.
\end{itemize}
\end{frame}




\begin{frame}[fragile]
\frametitle{Prerequisites}
\begin{itemize}
\item Scilab installed on your machine.\pause
\item Spoken Tutorial: Getting Started with Scilab.\pause
\item Spoken Tutorial: Vector Operations.\pause
\item I am using Windows 7 OS and Scilab 5.2.2 for demonstration.
\end{itemize}
\end{frame}








\begin{frame}[fragile]
\frametitle{Exercise 1}
If \textsl{A = $\left[
               \begin{array}{ccc}
                 1 & -1 & 0  \\ \\
                 2 & 3 & 1 \\ \\
                 4 & 1 & 5
              \end{array}
             \right]$ }\\
\end{frame}

\begin{frame}[fragile]
\frametitle{Exercise 1}
\begin{itemize}
\item Find A(:,:)
\item Extract the second column of A
\item Determine the determinant and eigenvalues of the matrix, A^2+2*A.
\end{itemize}
\end{frame}

\begin{frame}[fragile]
\frametitle{Linear Systems}
\begin{itemize}
\item One of the important sets of operations a user carries out on matrices are elementary  row or column operations.\pause
\item They involve executing row operations on a matrix to make entries below a non-zero number, zero.

\end{itemize}
\end{frame}

\begin{frame}[fragile]
\frametitle{Linear Equations}
Lets solve the following set of linear equations: \\
x1 + 2x2 - x3 = 1 \\
-2x1 - 6x2 + 4x3 = - 2 \\
-x1 - 3x2 + 3x3 = 1 \\
\end{frame}

\begin{frame}[fragile]
\frametitle{Exercise 2}
\begin{itemize}
\item Define a 3x3 matrix A with all elements equal to 1. \\ Multiply 1st and 2nd row with scalars, 3 and 4 respectively, and determine the determinant of the resultant matrix. 
\end{itemize}
\end{frame}

\begin{frame}[fragile]
\frametitle{Exercise 2}
\begin{itemize}
\item Represent the following linear system as a matrix equation. Solve the system using the inverse method:\\
\quad \quad x + y + 2z - w = 3 \\
\quad \quad 2x + 5y - z - 9w = -3 \\
\quad \quad 2x + y - z + 3w = -11 \\
\quad \quad x - 3y + 2z + 7w = -5 \\
\end{itemize}
\end{frame}

\begin{frame}[fragile]
\frametitle{Exercise 2}
a) Try solving the above system using the backslash method.\\
b) Verify the solution of part (a).\\
\end{frame}

\begin{frame}[fragile]
\frametitle{Exercise 2}
If \textsl{A = $\left[
               \begin{array}{ccc}
                 2 & 3 & 1  \\ \\
                 4 & 6 & 5 \\ \\
                 1 & 3 & 6
              \end{array}
             \right]$ }\\ \\ 
Use a suitable sequence of row operations on A to bring A to upper triangular form.

\end{frame}

\begin{frame}[fragile]
\frametitle{Summary}
In this tutorial, we have learnt to,
\begin{itemize}
\item Access the element of the matrix using colon(:).
\item Calculate the inverse of matrix using "inv" command or by black slash.\pause
\item Calculate the determinant of a matrix using "det()" command.\pause
\item Calculate the eigenvalues of a matrix using "spec()" command.
\end{itemize}
\end{frame}




\begin{frame}[fragile]
\frametitle{Summary continue..}
\begin{itemize}

\item Define a matrix having all elements one, Null matrix, Identity matrix
and a matrix with random elements by using functions ones(), zeros(), eye(), rand() respectively.\pause
\item Solve the systems of linear equations.
\end{itemize}
\end{frame}

\begin{frame}[fragile]
\frametitle{Acknowledgement}
\begin{itemize}
\item This spoken tutorial has been created by the Free and Open Source Software in Science and Engineering Education(FOSSEE).\pause
\item More information on the FOSSEE project could be obtained from {\color{magenta}http://fossee.in} or {\color{magenta}http://scilab.in}
\end{itemize}
\end{frame}


\begin{frame}[fragile]
\frametitle{Acknowledgement}
\begin{itemize}
\item Supported by the National Mission on Eduction through ICT, MHRD, Government of India.\pause
\item For more information, visit: \\
	{\color{magenta}http://spoken-tutorial.org/NMEICT-Intro}

\end{itemize}
\end{frame}
\end{document}


If \textsl{A = $\left[
               \begin{array}{ccc}
                 1 & -1 & 1  \\ \\
                 -1 & 1 & 1 \\ \\
                 1 & 1 & -1
              \end{array}
             \right]$ }\\
\begin{itemize}
\item Try det(A),$ A^2 , A^3$ and Eigen-
values of A (from the previous ques-
tion).Also multiply A by an identity matrix
of the same size.
\end{itemize}
\end{frame}


\end{document}

